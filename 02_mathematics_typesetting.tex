\section{Mathematical Typesetting}

\begin{frame}[fragile]{Inline vs. Display Math}
    \begin{columns}
        \begin{column}{0.5\textwidth}
            \begin{lstlisting}
% Inline math (within text):
Einstein's equation $E = mc^2$ 
revolutionized physics.

The quadratic formula is 
$x = \frac{-b \pm \sqrt{b^2 - 4ac}}{2a}$.

% Display math (centered, set apart):
\[
E = mc^2
\]

\begin{equation}
    x = \frac{-b \pm \sqrt{b^2 - 4ac}}{2a}
\end{equation}
            \end{lstlisting}
        \end{column}
        
        \begin{column}{0.5\textwidth}
            \textbf{Inline math:}
            \begin{itemize}
                \item Inside text flow: $E = mc^2$
                \item Uses \texttt{\$...\$} or \texttt{\textbackslash(..\\textbackslash)} delimiters
                \item Less vertical space
                \item Avoids line breaks
            \end{itemize}
            
            \textbf{Display math:}
            \begin{itemize}
                \item Set apart from text
                \item Centered on its own line
                \item \texttt{\textbackslash[...\textbackslash]} (unnumbered)
                \item \texttt{equation} environment (numbered)
            \end{itemize}
        \end{column}
    \end{columns}
    
    \begin{warning}
        Never use \texttt{\$\$...\$\$} for display math in LaTeX (a TeX relic)
    \end{warning}
\end{frame}

\begin{frame}[fragile]{Basic Math Syntax}
    \begin{columns}
        \begin{column}{0.5\textwidth}
            \begin{lstlisting}
% Subscripts and superscripts
$x^2$         % superscript
$x_i$         % subscript
$x_i^2$       % both
$x^{2n}$      % multi-character
$x_{i,j}$     % multi-character
$x^2_i$       % both

% Fractions
$\frac{1}{2}$           % standard
$\frac{\partial f}{\partial x}$  % complex
$\tfrac{1}{2}$          % text-style

% Square roots
$\sqrt{x}$              % square root
$\sqrt[3]{x}$           % cube root
            \end{lstlisting}
        \end{column}
        
        \begin{column}{0.5\textwidth}
            \textbf{Subscripts and superscripts:}
            \begin{itemize}
                \item $x^2$ (superscript)
                \item $x_i$ (subscript)
                \item $x_i^2$ (both)
                \item $x^{2n}$ (multi-character superscript)
                \item $x_{i,j}$ (multi-character subscript)
            \end{itemize}
            
            \textbf{Fractions:}
            \begin{itemize}
                \item $\frac{1}{2}$ (standard)
                \item $\frac{\partial f}{\partial x}$ (complex)
                \item $\tfrac{1}{2}$ (text-style)
            \end{itemize}
            
            \textbf{Square roots:}
            \begin{itemize}
                \item $\sqrt{x}$ (square root)
                \item $\sqrt[3]{x}$ (cube root)
            \end{itemize}
        \end{column}
    \end{columns}
\end{frame}

\begin{frame}[fragile]{Greek Letters and Special Symbols}
    \begin{columns}
        \begin{column}{0.5\textwidth}
            \begin{lstlisting}
% Lowercase Greek letters
$\alpha, \beta, \gamma, \delta$
$\epsilon, \varepsilon, \zeta, \eta$
$\theta, \vartheta, \iota, \kappa$
$\lambda, \mu, \nu, \xi$
$o, \pi, \varpi, \rho, \varrho$
$\sigma, \varsigma, \tau, \upsilon$
$\phi, \varphi, \chi, \psi, \omega$

% Uppercase Greek letters
$\Gamma, \Delta, \Theta, \Lambda$
$\Xi, \Pi, \Sigma, \Upsilon$
$\Phi, \Psi, \Omega$

% Special symbols
$\infty, \partial, \nabla, \wp$
$\emptyset, \forall, \exists$
$\in, \notin, \subset, \supset$
$\cup, \cap, \setminus$
            \end{lstlisting}
        \end{column}
        
        \begin{column}{0.5\textwidth}
            \textbf{Lowercase Greek:}
            \begin{center}
$\alpha, \beta, \gamma, \delta$\\
$\epsilon, \varepsilon, \zeta, \eta$\\
$\theta, \vartheta, \iota, \kappa$\\
$\lambda, \mu, \nu, \xi$\\
$o, \pi, \varpi, \rho, \varrho$\\
$\sigma, \varsigma, \tau, \upsilon$\\
$\phi, \varphi, \chi, \psi, \omega$
            \end{center}
            
            \textbf{Uppercase Greek:}
            \begin{center}
$\Gamma, \Delta, \Theta, \Lambda$\\
$\Xi, \Pi, \Sigma, \Upsilon$\\
$\Phi, \Psi, \Omega$
            \end{center}
            
            \textbf{Special symbols:}
            \begin{center}
$\infty, \partial, \nabla, \wp$\\
$\emptyset, \forall, \exists$\\
$\in, \notin, \subset, \supset$\\
$\cup, \cap, \setminus$
            \end{center}
        \end{column}
    \end{columns}
\end{frame}

\begin{frame}[fragile]{Mathematical Operators and Relations}
    \begin{columns}
        \begin{column}{0.5\textwidth}
\begin{lstlisting}
% Binary relations
$=, \ne, \approx, \sim, \cong$
$<, \le, >, \ge$
$\ll, \gg, \prec, \succ$

% Binary operators
$+, -, \times, \div, \pm, \mp$
$\cdot, \ast, \star, \circ, \bullet$
$\cup, \cap, \setminus, \wedge, \vee$

% Large operators
$\sum_{i=1}^{n} i = \frac{n(n+1)}{2}$

$\prod_{i=1}^{n} i = n!$

$\int_{a}^{b} f(x) \, dx$

$\oint_C f(z) \, dz = 2\pi i$

$\lim_{x \to 0} \frac{\sin x}{x} = 1$
\end{lstlisting}
        \end{column}
        
        \begin{column}{0.5\textwidth}
            \textbf{Binary relations:}
            \begin{center}
                $=, \ne, \approx, \sim, \cong$\\
                $<, \le, >, \ge$\\
                $\ll, \gg, \prec, \succ$
            \end{center}
            
            \textbf{Binary operators:}
            \begin{center}
                $+, -, \times, \div, \pm, \mp$\\
                $\cdot, \ast, \star, \circ, \bullet$\\
                $\cup, \cap, \setminus, \wedge, \vee$
            \end{center}
            
            \textbf{Large operators:}
            \begin{align*}
                \sum_{i=1}^{n} i &= \frac{n(n+1)}{2}\\
                \prod_{i=1}^{n} i &= n!\\
                \int_{a}^{b} f(x) \, dx\\
                \oint_C f(z) \, dz &= 2\pi i\\
                \lim_{x \to 0} \frac{\sin x}{x} &= 1
            \end{align*}
        \end{column}
    \end{columns}
\end{frame}

\begin{frame}[fragile]{Function Names}
    \begin{columns}
        \begin{column}{0.5\textwidth}
            \begin{lstlisting}
                % Wrong way (italicized)
                $sin(x)$
                $log(x)$
                $lim_{x \to 0} f(x)$
                
                % Correct way (upright)
                $\sin(x)$
                $\log(x)$
                $\lim_{x \to 0} f(x)$
                
                % Common function names
                $\sin(x), \cos(x), \tan(x)$
                $\arcsin(x), \arccos(x), \arctan(x)$
                $\exp(x), \ln(x), \log_{10}(x)$
                $\min(x,y), \max(x,y)$
                $\gcd(a,b), \det(A), \dim(V)$
            \end{lstlisting}
        \end{column}
        
        \begin{column}{0.5\textwidth}
            \textbf{Wrong (italicized):}
            \begin{itemize}
                \item $sin(x)$
                \item $log(x)$
                \item $lim_{x \to 0} f(x)$
            \end{itemize}
            
            \textbf{Correct (upright):}
            \begin{itemize}
                \item $\sin(x)$
                \item $\log(x)$
                \item $\lim_{x \to 0} f(x)$
            \end{itemize}
            
            \textbf{Common function names:}
            \begin{itemize}
                \item $\sin(x), \cos(x), \tan(x)$
                \item $\arcsin(x), \arccos(x), \arctan(x)$
                \item $\exp(x), \ln(x), \log_{10}(x)$
                \item $\min(x,y), \max(x,y)$
                \item $\gcd(a,b), \det(A), \dim(V)$
            \end{itemize}
        \end{column}
    \end{columns}
    
    \begin{tip}
        Always use the backslash command for function names
    \end{tip}
\end{frame}

\begin{frame}[fragile]{Delimiters (Brackets and Parentheses)}
    \begin{columns}
        \begin{column}{0.5\textwidth}
            \begin{lstlisting}
                % Fixed size
                $(x)$
                $[x]$
                $\{x\}$
                $|x|$
                $\|x\|$
                $\langle x \rangle$
                $\lceil x \rceil$
                $\lfloor x \rfloor$
                
                % Auto-scaling
                $\left( \frac{1}{1-x^2} \right)$
                
                $\left[ \sum_{i=1}^{n} a_i \right]$
                
                $\left\{ \frac{n(n+1)}{2} \right\}$
                
                $\left| \det(A) \right|$
            \end{lstlisting}
        \end{column}
        
        \begin{column}{0.5\textwidth}
            \textbf{Fixed size:}
            \begin{center}
                $(x)$ - parentheses\\
                $[x]$ - square brackets\\
                $\{x\}$ - curly braces\\
                $|x|$ - absolute value\\
                $\|x\|$ - norm\\
                $\langle x \rangle$ - angle brackets\\
                $\lceil x \rceil$ - ceiling\\
                $\lfloor x \rfloor$ - floor
            \end{center}
            
            \textbf{Auto-scaling:}
            \begin{center}
                $\left( \frac{1}{1-x^2} \right)$\\[5pt]
                $\left[ \sum_{i=1}^{n} a_i \right]$\\[5pt]
                $\left\{ \frac{n(n+1)}{2} \right\}$\\[5pt]
                $\left| \det(A) \right|$
            \end{center}
        \end{column}
    \end{columns}
    
    \begin{tip}
        Always use \texttt{\textbackslash left} and \texttt{\textbackslash right} for complex expressions
    \end{tip}
\end{frame}

\begin{frame}[fragile]{Matrices}
    \begin{columns}
        \begin{column}{0.5\textwidth}
            \begin{lstlisting}
                % Matrix without delimiters
                \begin{matrix}
                    a & b \\
                    c & d
                \end{matrix}
                
                % Matrix with parentheses
                \begin{pmatrix}
                    a & b \\
                    c & d
                \end{pmatrix}
                
                % Matrix with brackets
                \begin{bmatrix}
                    a & b \\
                    c & d
                \end{bmatrix}
                
                % Matrix with braces
                \begin{Bmatrix}
                    a & b \\
                    c & d
                \end{Bmatrix}
                
                % Matrix with vertical bars
                \begin{vmatrix}
                    a & b \\
                    c & d
                \end{vmatrix}
            \end{lstlisting}
        \end{column}
        
        \begin{column}{0.5\textwidth}
            \begin{center}
                $\begin{matrix} a & b \\ c & d \end{matrix}$ - no delimiters\\[10pt]
                
                $\begin{pmatrix} a & b \\ c & d \end{pmatrix}$ - parentheses\\[10pt]
                
                $\begin{bmatrix} a & b \\ c & d \end{bmatrix}$ - brackets\\[10pt]
                
                $\begin{Bmatrix} a & b \\ c & d \end{Bmatrix}$ - braces\\[10pt]
                
                $\begin{vmatrix} a & b \\ c & d \end{vmatrix}$ - determinant
            \end{center}
            
            \begin{itemize}
                \item Use \texttt{\&} to separate columns
                \item Use \texttt{\\} for new rows
                \item Requires \texttt{amsmath} package
            \end{itemize}
        \end{column}
    \end{columns}
\end{frame}

\begin{frame}[fragile]{Equation Alignment}
    \begin{columns}
        \begin{column}{0.5\textwidth}
            \begin{lstlisting}
                % Multiple equations (numbered)
                \begin{align}
                    E &= mc^2 \\
                    F &= ma
                \end{align}
                
                % Multiple equations (unnumbered)
                \begin{align*}
                    (a+b)^2 &= (a+b)(a+b) \\
                    &= a(a+b) + b(a+b) \\
                    &= a^2 + ab + ba + b^2 \\
                    &= a^2 + 2ab + b^2
                \end{align*}
                
                % Align at multiple places
                \begin{align}
                    f(x) &= (x+a)(x+b) \\
                    &= x^2 + (a+b)x + ab \\
                    &= x^2 + cx + d
                \end{align}
            \end{lstlisting}
        \end{column}
        
        \begin{column}{0.5\textwidth}
            \textbf{Multiple equations (numbered):}
            \begin{align}
                E &= mc^2 \\
                F &= ma
            \end{align}
            
            \textbf{Multiple equations (unnumbered):}
            \begin{align*}
                (a+b)^2 &= (a+b)(a+b) \\
                &= a(a+b) + b(a+b) \\
                &= a^2 + ab + ba + b^2 \\
                &= a^2 + 2ab + b^2
            \end{align*}
            
            \begin{itemize}
                \item Alignment point is marked with \texttt{\&}
                \item Each line ends with \texttt{\\}
                \item \texttt{align*} for unnumbered equations
                \item Requires \texttt{amsmath} package
            \end{itemize}
        \end{column}
    \end{columns}
    
    \begin{tip}
        Always align equations at the relation symbol (=, <, etc.)
    \end{tip}
\end{frame}

\begin{frame}[fragile]{Cases and Multi-Line Expressions}
    \begin{columns}
        \begin{column}{0.5\textwidth}
            \begin{lstlisting}
                % Cases (piecewise functions)
                \begin{equation}
                    |x| = 
                    \begin{cases}
                        x & \text{if } x \geq 0 \\
                        -x & \text{if } x < 0
                    \end{cases}
                \end{equation}
                
                % Multi-line expression
                \begin{equation}
                    \begin{split}
                        (a+b)^3 = &(a+b)^2(a+b) \\
                        = &(a^2+2ab+b^2)(a+b) \\
                        = &a^3 + 2a^2b + ab^2 \\
                        &+ a^2b + 2ab^2 + b^3 \\
                        = &a^3 + 3a^2b + 3ab^2 + b^3
                    \end{split}
                \end{equation}
            \end{lstlisting}
        \end{column}
        
        \begin{column}{0.5\textwidth}
            \textbf{Cases (piecewise functions):}
            \begin{equation}
                |x| = 
                \begin{cases}
                    x & \text{if } x \geq 0 \\
                    -x & \text{if } x < 0
                \end{cases}
            \end{equation}
            
            \textbf{Multi-line expression:}
            \begin{equation}
                \begin{split}
                    (a+b)^3 = &(a+b)^2(a+b) \\
                    = &(a^2+2ab+b^2)(a+b) \\
                    = &a^3 + 2a^2b + ab^2 \\
                    &+ a^2b + 2ab^2 + b^3 \\
                    = &a^3 + 3a^2b + 3ab^2 + b^3
                \end{split}
            \end{equation}
            
            \begin{tip}
                Use \texttt{\textbackslash text\{...\}} for text within math mode
            \end{tip}
        \end{column}
    \end{columns}
\end{frame}

\begin{frame}[fragile]{Additional Math Environments}
    \begin{columns}
        \begin{column}{0.5\textwidth}
            \begin{lstlisting}
                % Gather - centered, multi-line
                \begin{gather}
                    a_1 = b_1 + c_1\\
                    a_2 = b_2 + c_2 + d_2
                \end{gather}
                
                % Numbered array
                \begin{array}{lcr}
                    left & center & right\\
                    x+y & x & y
                \end{array}
                
                % Multiline equation
                \begin{multline}
                    a + b + c + d + e + f \\
                    + i + j + k + l + m
                \end{multline}
                
                % Subequations (hierarchical numbering)
                \begin{subequations}
                    \begin{align}
                        a &= b + c\\
                        d &= e + f
                    \end{align}
                \end{subequations}
            \end{lstlisting}
        \end{column}
        
        \begin{column}{0.5\textwidth}
            \textbf{Gather} - centered equations:
            \begin{gather}
                a_1 = b_1 + c_1\\
                a_2 = b_2 + c_2 + d_2
            \end{gather}
            
            \textbf{Array} - tabular math:
            \begin{center}
                $\begin{array}{lcr}
                    \text{left} & \text{center} & \text{right}\\
                    x+y & x & y
                \end{array}$
            \end{center}
            
            \textbf{Multline} - break long equations:
            \begin{multline}
                a + b + c + d + e + f \\
                + i + j + k + l + m
            \end{multline}
            
            \textbf{Subequations} - hierarchical:
            \begin{subequations}
                \begin{align}
                    a &= b + c\\
                    d &= e + f
                \end{align}
            \end{subequations}
        \end{column}
    \end{columns}
\end{frame}

\begin{frame}{Exercise 2: Mathematical Typesetting}
    \begin{practice}
        \textbf{Create a document with these mathematical elements:}
        \begin{enumerate}
            \item A paragraph with inline math expressions
            \item Multiple display math equations (both numbered and unnumbered)
            \item An equation with fractions, superscripts, subscripts, and Greek letters
            \item A piecewise function using cases
            \item A matrix equation
            \item A multi-line derivation with proper alignment
        \end{enumerate}
    \end{practice}
    
    \begin{tip}
        \begin{itemize}
            \item Use \texttt{align} for multiple aligned equations
            \item Remember \texttt{\textbackslash left} and \texttt{\textbackslash right} for auto-scaling delimiters
            \item Test complex expressions separately before combining them
        \end{itemize}
    \end{tip}
\end{frame}
