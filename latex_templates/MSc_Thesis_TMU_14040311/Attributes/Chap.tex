%
%
%آیین‌نامه چاپ پایان‌نامه/رساله
%
%
%در این فایل، کافی است در ابتدا، عبارت‌های موجود در خط‌های 20، 24، 28، 32، 36، 48، 52، 56، 64 و 67 را پاک و بعد، اطلاعات خود را تایپ کنید. 
%در خط 64، برای وارد کردن تصویر امضای خود به شکل عکس و با فرمت jpg، در ابتدا فایل Signature.jpg را در پوشه Figures حذف و تصویر امضای خود را با همان نام در همان پوشه اضافه کنید. (در صورتی که فرمت عکس jpg نباشد، در خط 64، عبارت jpg را از Signature.jpg حذف و فرمت جدید را اضافه کنید.)
%
%
\begin{center}
\textbf{
آیین‌نامه چاپ پایان‌نامه (رساله)های دانشجویان دانشگاه تربیت مدرس
}
\end{center}
\thispagestyle{empty}
نظر به اینکه چاپ و انتشار پایان‌نامه (رساله)های تحصیلی دانشجویان دانشگاه تربیت مدرس، مبین بخشی از فعالیت‌های علمی-پژوهشی دانشگاه است بنابراین به منظور آگاهی و رعایت حقوق دانشگاه، دانش‌آموختگان این دانشگاه نسبت به رعایت موارد ذیل متعهد می‌شوند:

ماده 1- در صورت اقدام به چاپ پایان‌نامه (رساله)ی خود، مراتب را قبلاً به طور کتبی به «دفتر نشر آثار علمی» دانشگاه اطلاع دهد.

ماده 2- در صفحه سوم کتاب (پس از برگ شناسنامه) عبارت ذیل را چاپ کند: «کتاب حاضر، حاصل پایان‌نامه کارشناسی ارشد/رساله دکتری نگارنده در رشته 
\textbf{
نام رشته خود را وارد کنید
}
است که در سال
\textbf{
سال را وارد کنید
}
در دانشکده علوم ریاضی دانشگاه تربیت مدرس به راهنمایی سرکار خانم/جناب آقای دکتر 
\textbf{
نام استاد راهنمای خود را وارد کنید
},
مشاوره سرکار خانم/جناب آقای دکتر
\textbf{ 
نام استاد مشاور اول خود را وارد کنید
}
و مشاوره سرکار خانم/جناب آقای دکتر 
\textbf{
نام استاد مشاور دوم خود را وارد کنید
}
از آن دفاع شده است.» 

ماده 3- به منظور جبران بخشی از هزینه‌های انتشارات دانشگاه، تعداد یک درصد شمارگان کتاب (در هر نوبت چاپ) را به «دفتر نشر آثار علمی» دانشگاه اهداد کند. دانشگاه می‌تواند مازاد نیاز خود را به نفع مرکز نشر در معرض فروض قرار دهد.

ماده 4- در صورت عدم رعایت ماده 3، 50 درصد بهای شمارگان چاپ شده را به عنوان خسارت به دانشگاه تربیت مدرس، تأدیه کند.

ماده 5- دانشجو تعهد و قبول می‌کند در صورت خودداری از پرداخت بهای خسارت، دانشگاه می‌تواند خسارت مذکور را از طریق مراجع قضایی مطالبه و وصول کند؛ به علاوه به دانشگاه حق می\nf دهد به منظور استیفای حقوق خود، از طریق دادگاه، معادل وجه مذکور در ماده 4 را از محل توقیف کتاب‌های عرضه شده نگارنده برای فروش، تأمین نماید.

ماده 6- اینجانب 
\textbf{
نام خود را وارد کنید
}
دانشجوی رشته 
\textbf{
نام رشته خود را وارد کنید
}
مقطع
\textbf{
مقطع خود را وارد کنید
}
تعهد فوق و ضمانت اجرایی آن را قبول کرده، به آن ملتزم می‌شوم.  
\begin{flushleft}
\makebox[3cm]{
امضا:
}
\\
\includegraphics[scale=0.2]{Signature.jpg}
\\
تاریخ:
تاریخ را وارد کنید
\end{flushleft}