%
%
%فایل حاضر به درخواست جناب آقای دکتر ناصر گلستانی، معاون پژوهشی وقت دانشکده علوم ریاضی دانشگاه تربیت مدرس، و با کمک دوستان عزیز آقایان رضا یغمائیان و ایوب احمدی، دانشجویان دکتری دانشگاه تربیت مدرس ورودی سال 1399 تهیه شده است.
%
%
%در پوشه Pdf، چند فایل برای آموزش برنامه لاتک قرار داده شده است. اگر در هنگام تایپ کردن، سؤالی برایتان پیش آمد، می‌توانید به این فایل‌ها مراجعه کنید.
%
%
%قالب خام رساله دکتری، رشته های دانشکده علوم ریاضی، دانشگاه تربیت مدرس تهران.
%
%
%در فایل حاضر، بصورت پیش‌فرض گروه دانشجو "آمار کاربردی" و رشته آن آمار در نظر گرفته شده است. اگر دانشجوی دیگر گروه‌های علوم ریاضی هستید، در فایل Phd_Thesis_TMU.cls، خط 95، کلمات "آمار کاربردی" و "آمار" را پاک و گروه خود را وارد کنید. مثلا می توانید گروه ریاضی محض و رشته ریاضی بنویسید.
%
%
\documentclass[phd,12pt]{Phd_Thesis_TMU} 

%دستور زیر برای چاپ فایل خروجی بصورت پشت و رو است. (برای چاپ بصورت تک رو، عبارت twosidetrue را دستور پایین حذف و عبارت twosidefalse را جایگزین آن کنید)
\csname@twosidetrue\endcsname 

%بسته زیر برای نمایش و تحلیل تنظیمات صفحه‌بندی (layout) یک سند به کار می‌رود. (تنظیم ابعاد و ویژگی‌های مختلف صفحه، مانند حاشیه‌ها، عرض و ارتفاع متن، فاصله بین عنوان‌ها و سرفصل‌ها، و سایر ویژگی‌های مرتبط با چیدمان و صفحه‌بندی با این بسته است.(
\usepackage{layout}

%بسته زیر برای تنظیم متن با اندازه‌های دلخواه غیر از اندازه‌های معمول \tiny، \scriptsize، \small، و ... برای فونت‌ها است. 
\usepackage{anyfontsize}

 %بسته زیر برای استفاده از فایل‌هایی که به صورت عکس یا pdf است، می‌باشد.
\usepackage{graphicx} 

%بسته زیر برای استفاده از گراف‌ها و دیاگرام‌ها است.
\usepackage[all]{xy} 

%بسته زیر برای دو ستونه یا چند ستونه کردن متن است. (با این دستور می‌توان متن را تا حداکثر 10 ستون نوشت) 
\usepackage{multicol} 

%بسته زیر برای رسم دیاگرام‌های برداری (فلش‌دار) است.
\usepackage{pb-diagram}

%بسته زیر برای استفاده از موارد مربوط به فرمول‌های ریاضی مانند \eqref، \tag، و ...، و همچنین تعریف نمادهای جدید شبیه \sin، \lim، و ... است.
\usepackage{amsmath}

%بسته زیر یک افزونه برای بسته قدرتمند amsmath است که امکانات اضافی و بهبودیافته‌ای برای نوشتن فرمول‌های ریاضی و تنظیمات ریاضی ارائه می‌دهد. 
\usepackage{mathtools}

%بسته زیر برای استفاده از نمادها مانند نمادهای دوتایی \cup و \cap و ... است.
\usepackage{amssymb}

%بسته زیر برای تعریف محیط‌هایی شبیه قضیه (theorem) مانند تعریف (definition)، لم (lemma)، و ... است. (این کار با استفاده از دستور \newtheorem{} و \theoremstyle{} انجام می‌شود)
\usepackage{amsthm}

%بسته زیر برای استفاده از فونت‌های اسکریپتی ریاضی (script fonts) مانند مجموعه‌ها، فضاها، توابع و ... است که با دستور \mathscr استفاده می‌شود.
\usepackage{mathrsfs}

%بسته زیر برای استفاده از فونت‌های ریاضی بیشتر مانند نمادهای جدید، حروف درشت و ... است.
\usepackage{amsfonts}

%بسته زیر برای ترسیم خطوط نقطه‌چین و خط‌چین در جدول‌ها و ماتریس‌ها استفاده می‌شود.
\usepackage{arydshln}

%بسته زیر برای مدیریت و اعمال رنگ‌ها در متن، جداول، نمودارها و ... است.
\usepackage{xcolor}

%بسته زیر برای مدیریت و کنترل بهتر محل قرارگیری اشیاء شناور (floats) مانند تصاویر، جداول و دیاگرام‌ها استفاده می‌شود.
\usepackage{float}

%بسته زیر برای استفاده از محیط‌های بصورت لیست، یعنی enumerate (شماره‌دار)، itemize (بدون شماره) است.
\usepackage{enumitem}

%بسته زیر برای استفاده از فونت‌های ریاضی مانند \infty، \prime، \vdots، و ... است.
\usepackage{fdsymbol}

%بسته زیر برای ارجاع دادن در داخل متن به مرجع یا یک مطلب عنوان‌دار (مانند قضیه، تعریف، لم، و ...) در همان متن استفاده می‌شود.
\usepackage{hyperref}
\hypersetup{pagebackref=true,colorlinks=true,linkcolor=blue,citecolor=magenta,urlcolor=Plum}

%بسته زیر برای بهبود بخشیدن به ارجاع دادن بطور خودکار، بدون نیاز به وارد کردن دستی عنوان، در داخل متن به مرجع یا یک مطلب عنوان‌دار (مانند قضیه، تعریف، لم، و ...) در همان متن استفاده می‌شود.
\usepackage{cleveref}

%بسته زیر برای اضافه شدن خودکار کتابنامه، نمایه، فصل‌ها، و ... به فهرست است.
\usepackage[nottoc]{tocbibind}

%دستور زیر برای ایجاد مراجع جداگانه برای هر فصل است.
%\usepackage{chapterbib}

%بسته زیر برای تولید متن نمونه (dummy text) یا متن ساختگی استفاده می‌شود. (این بسته به شما امکان می‌دهد تا براحتی متن‌های آزمایشی را برای تست چیدمان، قالب‌بندی و طراحی در سند خود قرار دهید.(
\usepackage{lipsum}

%بسته زیر یک ابزار بسیار قدرتمند برای رسم اشکال، دیاگرام‌ها، نمودارها، و حتی تصاویر پیچیده است. 
\usepackage{tikz}

%بسته زیر برای قرار دادن متون پاراگرافی (به‌صورت چپ‌چین یا راست‌چین) در داخل اشکال هندسی رسم‌شده یا به ‌طور کلی در داخل محیط‌های گرافیکی، مانند آنچه در بسته tikz ایجاد می‌شود، استفاده می‌شود. هدف اصلی این بسته، امکان درج متون به‌صورت پاراگراف (نه فقط یک خط) در اشیاء گرافیکی است.
\usepackage{ptext}

%بسته زیر برای ایجاد و نمایش آرایه‌ها و ماتریس‌های بلوکی (block arrays and matrices) استفاده می‌شود.
\usepackage{blkarray}

%بسته زیر مربوط به مدیریت و فرمت‌دهی به بخش‌های ضمیمه است.
\usepackage{appendix}

%بسته زیر برای استفاده یک متن کنار شکل یا جدول است. 
\usepackage{wrapfig}

%بسته زیر برای تغییر و کنترل شمارشگرهای (counters) مختلف در سند به کار می‌رود. (این بسته به شما امکان می‌دهد که نحوه شماره‌گذاری عناصری مانند جداول، شکل‌ها، بخش‌ها، زیربخش‌ها، معادلات و سایر اجزای سند را به ‌صورت دلخواه تغییر دهید.(
\usepackage{chngcntr}

%این بسته برای تنظیم رمزگذاری ورودی، مانند ایجاد فاصله اولین کلمه از ابتدای خط در محیط‌های چپ‌چین مانند flushleft، است.
\usepackage[utf8]{inputenc}

%بسته زیر برای سفارشی ‌سازی عناوین بخش‌ها و زیربخش‌های سند استفاده می‌شود. (این بسته به شما اجازه می‌دهد تا نحوه نمایش و قالب‌بندی عناوین بخش‌ها (مانند \section، \subsection، و ...) را به دلخواه تغییر دهید و کنترل بیشتری بر روی ظاهر عناوین داشته باشید.(
\usepackage{titlesec} 
\titleformat{\chapter}[display] {\normalfont\sffamily\huge\bfseries\color{blue}} {\chaptertitlename\ \thechapter}{20pt}{\Huge}

\usepackage{multirow}

%دستور زیر برای نمایش دادن برچسب‌هایی (label) که در زمان تایپ استفاده کرده‌اید، در فایل pdf خروجی است.
%\usepackage{showkeys}

%دستور زیر برای استفاده از شکل‌ها از پوشه \Figures است.
\graphicspath{{Figures/}} 

%دستور زیر برای تنظیم فاصله بین دو خط در کل متن است.
\linespread{1.8}

%بسته زیر برای تایپ متن به زبان فارسی است.
\usepackage{xepersian} 

%دستور زیر برای تعیین فونت متن اصلی است.
\settextfont[Scale=1.1]{XB Niloofar} 

%دستور زیر برای تعریف فونت‌های خاص و استفاده از آنها در بخش‌های مدنظر در متن است. (برای استفاده از فونت IranNastaliq، قبل از متن مدنظر، \nastaliq را تایپ و با یک فاصله (space) متن خود را وارد کنید.( 
\defpersianfont\nastaliq[Scale=2]{IranNastaliq} 

%دستور زیر برای تعیین فونت اعداد و ارقام در متن اصلی است.
\setdigitfont[Scale=1.1]{XB Yas}  

%برای راحتی کار در تایپ نماد مجموعه‌های معمول در ریاضی، مانند اعداد طبیعی، اعداد صحیح، اعداد گویا و ...، از عبارت‌هایی که در {} اول بعد از کلمه \newcommand آمده‌اند، استفاده کنید.  
\newcommand{\N}{\mathbb{N}} 
\newcommand{\Z}{\mathbb{Z}} 
\newcommand{\Q}{\mathbb{Q}} 
\newcommand{\R}{\mathbb{R}} 
\newcommand{\CC}{\mathbb{C}} 
\newcommand{\Prime}{\mathbb{P}} 

%دستورهای زیر برای نمایش دادن ترجمه عنوان‌های معمول در ریاضی، مانند تعریف، قضیه، گزاره، لم و ... هستند. در زمان تایپ، از عبارت‌هایی در {} اول بعد از کلمه \newtheorem آمده‌اند، استفاده کنید.  
\theoremstyle{definition} 
\newtheorem{definition}{تعریف}[section] 
\newtheorem{remark}{ملاحظه} 
\newtheorem*{notation}{نمادگذاری} 
\newtheorem*{conjecture}{حدس} 
\theoremstyle{theorem} 
\newtheorem{theorem}[definition]{قضیه} 
\newtheorem{lemma}[definition]{لم} 
\newtheorem{proposition}[definition]{گزاره} 
\newtheorem{corollary}[definition]{نتیجه} 

\theoremstyle{definition} 
\newtheorem{example}[definition]{مثال} 
\theoremstyle{remark} 

%دستور زیر برای وارد کردن نیم‌فاصله بصورت دستی است.
\newcommand*{\nf}{^^^^200c}

\begin{document}   

%دستور زیر برای تنظیم فاصله بین دو خط در بخش‌های خاص متن مانند محیط \begin{}-\end{} است.
\baselineskip=.75cm 

%دستور زیر برای شماره‌گذاری صفحات بعد از این دستور بصورت حروف الفبای فارسی، یعنی آ، ب، پ، و ... است.
\pagenumbering{harfi} 
%دستورهای دیگر برای شماره‌گذاری صفحات مذکور (برای استفاده، کلمه اول هر خط را متناسب با نیاز خود داخل {}، بجای کلمه harfi، قرار دهید):
%arabic --> 1, 2, 3, ...
%roman --> i, ii, iii, ...
%Roman --> I, II, III, ...
%alph --> a, b, c, ...
%Alph --> A, B, C, ... 

%دستور زیر برای وارد کردن مشخصات خود و پایان‌نامه به زبان فارسی است. برای واردن کردن اطلاعات شخصی و اطلاعات مربوط به رساله خود، وارد پوشه Attributes شوید و اطلاعات را در فایل FA-Attributes.tex وارد کنید.
%
%
%در این فایل، عنوان رساله و مشخصات خود را به فارسی، وارد کنید.
%
%
\university{
دانشگاه تربیت مدرس
}

\faculty{
دانشگاه تربیت مدرس
}

\department{
دانشکده علوم ریاضی
}

%در فایل حاضر، بصورت پیش‌فرض گروه دانشجو "آمار کاربردی" و رشته آن آمار در نظر گرفته شده است. اگر دانشجوی دیگر گروه‌های علوم ریاضی هستید، در فایل Phd_Thesis_TMU.cls، خط 95، کلمات "آمار کاربردی" و "آمار" را پاک و گروه خود را وارد کنید. مثلا می توانید گروه ریاضی محض و رشته ریاضی بنویسید.

\title{
عنوان را وارد کنید
}

\firstsupervisor{
نام استاد راهنما را وارد کنید
}

%\secondsupervisor{
%نام استاد راهنمای دوم را وارد کنید
%}

\firstadvisor{
نام استاد مشاور را وارد کنید
}

%\secondadvisor{
%نام استاد مشاور دوم را وارد کنید
%}

\name{
نام و نام خانوادگی خود را وارد کنید
}

\thesisdate{
ماه و سال را وارد کنید
}


\vtitle

\clearpage
 
\thispagestyle{empty}
\vspace*{2cm}
\centerline{{
\includegraphics[height=15cm]{Besmellah}
}}
 
\newpage

\thispagestyle{empty}
\vspace*{.1cm}
\centerline{{
\includegraphics[height=27cm]{Taid}
}} 

%برای وارد کردن مشخصات و امضاء خود در آیین‌نامه حق مالکیت، وارد پوشه Attributes شوید و اطلاعات خود را در فایل Malekiat.tex وارد کنید.
%
%
%آیین‌نامه حق مالکیت مادی و معنوی
%
%
%در این فایل، کافی است در ابتدا، عبارت‌های موجود در خط‌های 33، 37، 41، 45، 57 و 60 را پاک و بعد، اطلاعات خود را تایپ کنید. 
%در خط 57، برای وارد کردن تصویر امضای خود به شکل عکس و با فرمت jpg، در ابتدا فایل Signature.jpg را در پوشه Figures حذف و تصویر امضای خود را با همان نام در همان پوشه اضافه کنید. (در صورتی که فرمت عکس jpg نباشد، در خط 57، عبارت jpg را از Signature.jpg حذف و فرمت جدید را اضافه کنید)
%
%
\begin{center}
\textbf{
آیین‌نامه حق مالکیت مادی و معنوی در مورد نتایج پژوهش‌های علمی دانشگاه تربیت مدرس
}
\end{center}
\thispagestyle{empty}
%دستور زیر برای تغییر اندازه کل متنی است که بعد از این دستور می‌آید.
\fontsize{3.4mm}{3.4mm}\selectfont \textbf{
مقدمه:
}
با عنایت به سیاست‌های پژوهشی و فناوری دانشگاه در راستای تحقق عدالت و کرامت انسان‌ها که لازمه شکوفایی علمی و فنی است و رعایت حقوق مادی و معنوی دانشگاه و پژوهشگران، لازم است اعضای هیأت علمی، دانشجویان، دانش‌آموختگان و دیگر همکاران طرح، در مورد نتایج پژوهش‌های علمی که تحت عناوین پایان‌نامه، رساله و طرح‌های تحقیقاتی با هماهنگی دانشگاه انجام شده است، موارد زیر را رعایت نمایند:

ماده 1- حق نشر و تکثیر پایان‌نامه/رساله و درآمدهای حاصل از آنها متعلق به دانشگاه می‌باشد ولی حقوق معنوی پدیدآورندگان محفوظ خواهد بود.

ماده 2- انتشار مقاله یا مقالات مستخرج از پایان‌نامه/رساله به صورت چاپ در نشریات علمی و یا ارائه در مجامع علمی باید به نام دانشگاه بوده و با تأیید استاد راهنمای اصلی، یکی از اساتید راهنما، مشاور و یا دانشجو مسئول مکاتبات مقاله باشد. ولی مسئولیت علمی مقاله مستخرج از پایان‌نامه/رساله به عهده اساتید راهنما و دانشجو می‌باشد. 

تبصره: در مقالاتی که پس از دانش‌آموختگی به صورت ترکیبی از اطلاعات جدید و نتایج حاصل از پایان‌نامه/رساله منتشر می‌شود نیز باید نام دانشگاه درج شود.

ماده 3- انتشار کتاب، نرم‌افزار و یا آثار ویژه (اثری هنری مانند فیلم، عکس، نقاشی و نمایش‌نامه) حاصل از نتایج پایان‌نامه/رساله و تمامی طرح‌های تحقیقاتی کلیه واحدهای دانشگاه اعم از دانشکده‌ها، مراکز تحقیقاتی، پژوهشکده‌ها، پارک علم و فناوری و دیگر واحدها باید با مجوز کتبی صادره از معاونت پژوهشی دانشگاه و بر اساس آیین‌نامه‌ها مصوب انجام شود.

ماده 4- ثبت اختراع و تدوین دانش فنی و یا ارائه یافته‌ها در جشنواره‌های ملی، منطقه‌ای و بین‌المللی حاصل نتایج مستخرج از پایان‌نامه/رساله و تمامی طرح‌های تحقیقاتی دانشگاه باید با هماهنگی استاد راهنما یا مجری طرح از طریق معاونت پژوهشی دانشگاه انجام گیرد.

ماده 5- این آیین‌نامه در 5 ماده و یک تبصره در تاریخ 01/04/1387 در شورای پژوهشی و در تاریخ 23/04/1387 در هیأت رئیسه دانشگاه به تأیید رسید و در جلسه مورخ 15/07/1387 شورای دانشگاه به تصویب رسیده و از تاریخ تصویب در شورای دانشگاه لازم‌الاجرا است.

«اینجانب 
\textbf{
نام خود را وارد کنید
}
دانشجوی رشته 
\textbf{
نام رشته خود را وارد کنید
}
ورودی سال تحصیلی 
\textbf{
سال ورود خود را وارد کنید
}
مقطع 
\textbf{
مقطع خود را وارد کنید
}
دانشکده 
\textbf{
علوم ریاضی 
}
متعهد می‌شوم کلیه نکات مندرج در آیین‌نامه حق مالکیت مادی و معنوی در مورد نتایج پژوهش‌های علمی دانشگاه تربیت مدرس را در انتشار یافته‌های علمی مستخرج از پایان‌نامه/رساله تحصیلی خود رعایت نمایم. در صورت تخلف از مفاد آیین‌نامه فوق‌الاشعار به دانشگاه وکالت و نمایندگی می‌دهم که از طرف اینجانب نسبت به لغو امتیاز اختراع به نام بنده و یا هر گونه امتیاز دیگر و تغییر آن به نام دانشگاه اقدام نماید. ضمناً نسبت به جبران فوری ضرر و زیان حاصله بر اساس برآورد دانشگاه اقدام خواهم نمود و بدین وسیله حق هر گونه اعتراض را از خود سلب نمودم.» 
\begin{flushleft}
\makebox[3cm]{
امضا:
}
\\
\includegraphics[scale=0.2]{Signature.jpg}
\\
تاریخ:
تاریخ را وارد کنید
\end{flushleft}

\newpage

%دستور زیر برای وارد کردن مشخصات و امضاء خود در آیین‌نامه چاپ است. (پوشه Attributes، فایل (Chap.tex
%
%
%آیین‌نامه چاپ پایان‌نامه/رساله
%
%
%در این فایل، کافی است در ابتدا، عبارت‌های موجود در خط‌های 20، 24، 28، 32، 36، 48، 52، 56، 64 و 67 را پاک و بعد، اطلاعات خود را تایپ کنید. 
%در خط 64، برای وارد کردن تصویر امضای خود به شکل عکس و با فرمت jpg، در ابتدا فایل Signature.jpg را در پوشه Figures حذف و تصویر امضای خود را با همان نام در همان پوشه اضافه کنید. (در صورتی که فرمت عکس jpg نباشد، در خط 64، عبارت jpg را از Signature.jpg حذف و فرمت جدید را اضافه کنید.)
%
%
\begin{center}
\textbf{
آیین‌نامه چاپ پایان‌نامه (رساله)های دانشجویان دانشگاه تربیت مدرس
}
\end{center}
\thispagestyle{empty}
نظر به اینکه چاپ و انتشار پایان‌نامه (رساله)های تحصیلی دانشجویان دانشگاه تربیت مدرس، مبین بخشی از فعالیت‌های علمی-پژوهشی دانشگاه است بنابراین به منظور آگاهی و رعایت حقوق دانشگاه، دانش‌آموختگان این دانشگاه نسبت به رعایت موارد ذیل متعهد می‌شوند:

ماده 1- در صورت اقدام به چاپ پایان‌نامه (رساله)ی خود، مراتب را قبلاً به طور کتبی به «دفتر نشر آثار علمی» دانشگاه اطلاع دهد.

ماده 2- در صفحه سوم کتاب (پس از برگ شناسنامه) عبارت ذیل را چاپ کند: «کتاب حاضر، حاصل پایان‌نامه کارشناسی ارشد/رساله دکتری نگارنده در رشته 
\textbf{
نام رشته خود را وارد کنید
}
است که در سال
\textbf{
سال را وارد کنید
}
در دانشکده علوم ریاضی دانشگاه تربیت مدرس به راهنمایی سرکار خانم/جناب آقای دکتر 
\textbf{
نام استاد راهنمای خود را وارد کنید
},
مشاوره سرکار خانم/جناب آقای دکتر
\textbf{ 
نام استاد مشاور اول خود را وارد کنید
}
و مشاوره سرکار خانم/جناب آقای دکتر 
\textbf{
نام استاد مشاور دوم خود را وارد کنید
}
از آن دفاع شده است.» 

ماده 3- به منظور جبران بخشی از هزینه‌های انتشارات دانشگاه، تعداد یک درصد شمارگان کتاب (در هر نوبت چاپ) را به «دفتر نشر آثار علمی» دانشگاه اهداد کند. دانشگاه می‌تواند مازاد نیاز خود را به نفع مرکز نشر در معرض فروض قرار دهد.

ماده 4- در صورت عدم رعایت ماده 3، 50 درصد بهای شمارگان چاپ شده را به عنوان خسارت به دانشگاه تربیت مدرس، تأدیه کند.

ماده 5- دانشجو تعهد و قبول می‌کند در صورت خودداری از پرداخت بهای خسارت، دانشگاه می‌تواند خسارت مذکور را از طریق مراجع قضایی مطالبه و وصول کند؛ به علاوه به دانشگاه حق می\nf دهد به منظور استیفای حقوق خود، از طریق دادگاه، معادل وجه مذکور در ماده 4 را از محل توقیف کتاب‌های عرضه شده نگارنده برای فروش، تأمین نماید.

ماده 6- اینجانب 
\textbf{
نام خود را وارد کنید
}
دانشجوی رشته 
\textbf{
نام رشته خود را وارد کنید
}
مقطع
\textbf{
مقطع خود را وارد کنید
}
تعهد فوق و ضمانت اجرایی آن را قبول کرده، به آن ملتزم می‌شوم.  
\begin{flushleft}
\makebox[3cm]{
امضا:
}
\\
\includegraphics[scale=0.2]{Signature.jpg}
\\
تاریخ:
تاریخ را وارد کنید
\end{flushleft}

\newpage

%دستور زیر برای وارد کردن متن تقدیم است. (پوشه Attributes، فایل (Taqdim.tex
%
%
%در این فایل‏، رساله خود را تقدیم کنید.
%
%
‎\begin{large}‎
‎\nastaliq{‎
تقدیم به
}
‎\end{large}‎
‎\thispagestyle{empty}‎
‎\vspace{1cm}‎
‎\\‎

‎\begin{large}‎
رساله خود را تقدیم کنید.
‎\end{large}

\newpage

%دستور زیر برای وارد کردن متن سپاس است. (پوشه Attributes، فایل (Sepas.tex
%
%
%در این فایل متن سپاس‌گزاری پایان‌نامه خود را وارد کنید.
%
%
\begin{large}

\nastaliq{
سپاس‌گزاری
}
\end{large}

\thispagestyle{empty}

\vspace{1cm}
\begin{large}
متن سپاس خود را وارد کنید.
\end{large}

%دستور زیر برای وارد کردن چکیده فارسی است. (پوشه Contents، فایل (FA-Abstract.tex
%
%
%در این فایل، چکیده فارسی رساله خود را وارد کنید.
%
%
\chapter*{
\centerline{
چکیده
}}
\addcontentsline{toc}{chapter}{چکیده}

متن چکیده خود را وارد کنید.

\vspace{0.5cm}

\textbf{
واژگان کلیدی:
}
واژگان کلیدی خود را وارد کنید.








\newpage

%دستور زیر برای وارد کردن مسئولیت اجتماعی و اثرگذاری رساله است. (پوشه Attributes، فایل (Asar.tex
%
%
%در این فایل مسئولیت اجتماعی و اثرگذاری رساله خود را وارد کنید. برای نوشتن متن خود، باید اشاره کنید پژوهشی که انجام داده‌اید چه تأثیری در محیط خارج از دانشگاه گذاشته و یا خواهد گذاشت؟ نیز نهاد، سازمان یا صنعت مورد نظر، تأثیر صورت پذیرفته را بیان کنید.
%
%
\centerline{\textbf{\LARGE{
مسئولیت اجتماعی و اثرگذاری پژوهش
}}}
\thispagestyle{empty}
\vspace{1cm}

متن اثرگذاری خود را وارد کنید.

%دستور زیر برای وارد شدن فهرست فصل‌ها، بطور خودکار، است.
\tableofcontents
%اگر شکل یا جدول ندارید یا نمی خواهید فهرست آنها در  در خروجی ظاهر شود، جلوی دستورهای زیر درصد قرار دهید
\listoftables
\listoffigures
\clearpage 

%دستور زیر برای وارد شدن فهرست شکل‌ها، بطور خودکار، است. (برای استفاده از این دستور، علامت % را از ابتدای دستورهای موجود در خط بعد و خط بعد از آن، پاک کنید.(
%\listoffigures
%\clearpage 

%دستور زیر برای وارد شدن فهرست جدول‌ها، بطور خودکار، است. (برای استفاده از این دستور، علامت % را از ابتدای دستورهای موجود در خط بعد و خط بعد از آن، پاک کنید).
%\listoftables
%\clearpage 

%دستور زیر برای شماره‌گذاری صفحات بعد از این دستور، به صورت عدد، یعنی 1، 2، 3، ... است.
\pagenumbering{arabic} 

%دستور زیر برای وارد کردن پیش‌گفتار است. (پوشه Contents، فایل (Introduction.tex
%
%
%در این فایل، پیش‌گفتار رساله خود را وارد کنید.
%
%
\chapter*{
پیش‌گفتار
}\label{chp_int}
\addcontentsline{toc}{chapter}{پیش‌گفتار}

پیش‌گفتار خود را وارد کنید. 

%دستور زیر برای وارد کردن فصل اول است. (پوشه Contents، فایل (Chapter1.tex
\include{Contents/chapter1} 

%دستور زیر برای وارد کردن فصل دوم است. (پوشه Contents، فایل (Chapter2.tex
\include{Contents/chapter2} 

%دستور زیر برای مشخص کردن شروع بخش ضمائم در یک سند استفاده می‌شود. (هنگامی که این دستور را به کار می‌برید، بخش‌های بعدی به عنوان ضمیمه در نظر گرفته می‌شوند.(
\appendix 

%دستور زیر برای وارد کردن فصل اول قسمت ضمیمه است. (پوشه Contents، فایل (Appendix1.tex
%
%
%در این فایل، فصل پیوست آ رساله خود را وارد کنید.
%
%
\chapter{
عنوان پیوست آ را وارد کنید
}\label{chapp1}

مطالب پیوست آ را وارد کنید.

\section{
عنوان بخش اول را وارد کنید
}\label{secapp11}

مطالب بخش اول را وارد کنید.

\section{
عنوان بخش دوم را وارد کنید
}\label{secapp12}

مطالب بخش دوم را وارد کنید.

 

%دستور زیر برای وارد کردن مراجع است. (پوشه Conductors، فایل (Referencec.tex
%
%
%در این فایل، مراجع رساله خود را وارد کنید.
%
%
% دستوری برای کوچک کردن اندازه فونت‌ها 
\small
% شروع محیط مراجع 

\begin{thebibliography}{99}

\bibitem{ag85} 
انجمن ریاضی ایران با همکاری گروه ریاضی و آمار مرکز نشر دانشگاهی، 
\emph{%
واژه‌نامه ریاضی و آمار%
},
مرکز نشر دانشگاهی، 1385.

\bibitem{go02} 

گوروئی، لیلا، 
\emph{
رتبه توپولوژیک و پیچیدگی زیرشیفت‌ها،
}
پایان‌نامه کارشناسی ارشد، دانشگاه تربیت مدرس، دانشکده علوم ریاضی، 1402.

\begin{LTRitems}
\resetlatinfont

\bibitem{bs04} 
M. Brin and G. Stuck, \emph{Introduction to Dynamical systems}, 
Cambridge University Press, 2004.

\bibitem{hps92} 
R. H. Herman, I. F. Putnam, and C. F. Skau, \emph{Ordered Bratteli diagrams, 
dimension groups and topological dynamics}, Int. J. Math. \textbf{3}, no. 6, 
827--864, 1992.

\end{LTRitems}



\end{thebibliography} 

%دستور زیر برای تنظیم فاصله خط پایه (Baseline Skip) بین خطوط یک متن استفاده می‌شود. (این فاصله به معنای فاصله عمودی بین خطوط متوالی است که از خط پایه یک خط تا خط پایه خط بعدی اندازه‌گیری می‌شود.(
\baselineskip=.75cm 

%دستور زیر برای وارد کردن واژگان فارسی به انگلیسی است. (پوشه Conductors، فایل (Dicfa2en.tex
%
%
%در این فایل، واژه‌های فارسی به انگلیسی پایان‌نامه خود را به طور دستی وارد کنید.
%
%
\chapter*{واژه‌نامه فارسی به انگلیسی}\markboth{واژه‌نامه فارسی به انگلیسی}{واژه‌نامه فارسی به انگلیسی}
\addcontentsline{toc}{chapter}{واژه‌نامه فارسی به انگلیسی}
\thispagestyle{empty}
{\LTR\setlength{\columnsep}{1cm} 
\begin{multicols}{2} 
\noindent
\persiangloss{
کلمه را وارد کنید
}{
Enter a word
}
\persiangloss{
کلمه را وارد کنید
}{
Enter a word
}
\persiangloss{
کلمه را وارد کنید
}{
Enter a word
}
\persiangloss{
کلمه را وارد کنید
}{
Enter a word
}
\persiangloss{
کلمه را وارد کنید
}{
Enter a word
}
\persiangloss{
کلمه را وارد کنید
}{
Enter a word
}
\persiangloss{
کلمه را وارد کنید
}{
Enter a word
}

\end{multicols} } 

%دستور زیر برای وارد کردن واژگان انگلیسی به فارسی است. (پوشه Conductors، فایل (Dicen2fa.tex
%
%
%در این فایل، واژه‌های انگلیسی به فارسی پایان‌نامه خود را به طور دستی وارد کنید.
%
%
\chapter*{واژه‌نامه  انگلیسی به  فارسی}\markboth{واژه‌نامه  انگلیسی به  فارسی}{واژه‌نامه  انگلیسی به  فارسی}
\addcontentsline{toc}{chapter}{واژه‌نامه  انگلیسی به  فارسی}
\thispagestyle{empty}
{\LTR\setlength{\columnsep}{1cm} 
\begin{multicols}{2} 
\noindent
\persiangloss{
کلمه را وارد کنید
}{
Enter a word
}
\persiangloss{
کلمه را وارد کنید
}{
Enter a word
}
\persiangloss{
کلمه را وارد کنید
}{
Enter a word
}
\persiangloss{
کلمه را وارد کنید
}{
Enter a word
}
\persiangloss{
کلمه را وارد کنید
}{
Enter a word
}
\persiangloss{
کلمه را وارد کنید
}{
Enter a word
}
\persiangloss{
کلمه را وارد کنید
}{
Enter a word
}

\end{multicols} } 

%دستور زیر برای چاپ و نمایش فهرست نمایه (index) است.
\printindex 

%دستور زیر برای وارد کردن چکیده انگلیسی است. (پوشه Contents، فایل (EN-Abstract.tex
%
%
%در این فایل، چکیده انگلیسی رساله خود را وارد کنید.
%
%
\chapter*{
\centerline{
Abstract
}}
\addcontentsline{toc}{chapter}{Abstract}

\begin{latin}
Enter your abstract.

\vspace{0.5cm}

\textbf{Keywords:}
Enter keywords.

\end{latin}







%دستور زیر برای وارد کردن مشخصات خود و پایان‌نامه به زبان انگلیسی است. (پوشه Attributes، فایل (EN-Attributes.tex
%
%
%در این فایل، عنوان پایان‌نامه و مشخصات خود را به انگلیسی، وارد کنید.
%
%
\baselineskip=.6cm

\begin{latin}

\latinuniversity{
Tarbiat Modares University
}

\latinfaculty{
Faculty of Mathematical Sciences
}

%بطور پیش فرض رشته  Statistics نوشته شده است. اگر رشته دیگری هستید، در خط 193 در فایل  Phd_Thesis_TMU.cls در همین پوشه بجای کلمه Statistics مثلا Mathematics بنویسید.

\latintitle{
Enter a title
}

\firstlatinsupervisor{
Enter supervision name
}

%\secondlatinsupervisor{
%Enter the second supervisor
%}

\firstlatinadvisor{
Enter advisor name
}

%\secondlatinadvisor{
%Enter the second advisor
%}

\latinname{
Enter your first and family name
}

\latinthesisdate{
Enter the month and year
}

\latinvtitle

\end{latin} 

%دستور زیر برای ارجاع دادن به صفحه آخر داخل متن است. (برای این کار در هر قسمت از متن، از دستور \ref{LastPage} استفاده کنید.(
\label{LastPage} 

\end{document} 